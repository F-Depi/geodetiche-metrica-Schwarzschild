\usepackage[utf8]{inputenc}

\usepackage{multirow,rotating}
\usepackage{amsmath}
%For mathematical formulas
\usepackage{xparse}
\usepackage{amssymb}
\usepackage{physics}
\usepackage{color}
\usepackage{hyperref}
\usepackage{tikz-cd}
\usepackage{array}
\usepackage{siunitx}
\usepackage{mathtools,nccmath}%
\usepackage{etoolbox, xparse} 
\usetheme{CambridgeUS}
\usecolortheme{dolphin}
\usepackage{tikz}
\usetikzlibrary{matrix}

% set colors
\definecolor{myNewColorA}{RGB}{158, 27,50}
\definecolor{myNewColorB}{RGB}{158, 27,50}
\definecolor{myNewColorC}{RGB}{158, 27,50} % {130,138,143}
\setbeamercolor*{palette primary}{bg=myNewColorC}
\setbeamercolor*{palette secondary}{bg=myNewColorB, fg = white}
\setbeamercolor*{palette tertiary}{bg=myNewColorA, fg = white}
\setbeamercolor*{titlelike}{fg=myNewColorA}
\setbeamercolor*{title}{bg=myNewColorA, fg = white}
\setbeamercolor*{item}{fg=myNewColorA}
\setbeamercolor*{caption name}{fg=myNewColorA}
\usefonttheme{professionalfonts}
\usepackage{natbib}
\usepackage{hyperref}
\usepackage[labelformat=empty]{caption}
\usepackage{braket}

%------------------------------------------------------------
% \titlegraphic{\includegraphics[height=0.75cm]{ua_eng_logo.png}} 

\titlegraphic{%
\includegraphics[width=3.0cm]{Figures/logo_rosso.png}
}

\setbeamerfont{title}{size=\large}
\setbeamerfont{subtitle}{size=\small}
\setbeamerfont{author}{size=\small}
\setbeamerfont{date}{size=\footnotesize}
\setbeamerfont{institute}{size=\footnotesize}
\title[Geodesics in Schwarzschild Metric]{Geodesics in Schwarzschild Metric}%title
%\subtitle{ }%%subtitle
\author[]{Candidate: Federico De Paoli\\ Supervisor: Prof. Albino Perego}%%authors

\institute[]{Bachelor degree in Physics, \\ University of Trento}
\date{September 18, 2024}




%% I'm defining two commands for the backup slides so there are not the slide numbers in the presentation (ex. 17/15)
\newcommand{\backupbegin}{
   \newcounter{finalframe}
   \setcounter{finalframe}{\value{framenumber}}
}
\newcommand{\backupend}{
   \setcounter{framenumber}{\value{finalframe}}
}


% Aggiunti da me

\newcommand{\Sh}[0]{Schwarzschild }
\newcommand{\Mi}[0]{Minkowski }
\usepackage{multimedia}
\usepackage{nicematrix}
\usepackage{tikz-3dplot}
\usepackage[overload]{empheq}
\usepackage{listings}


% Colors for the code
\definecolor{mygreen}{rgb}{0,0.6,0}
\definecolor{mygray}{rgb}{0.5,0.5,0.5}
\definecolor{mymauve}{rgb}{0.58,0,0.82}

% code style
\lstset{ 
  backgroundcolor=\color{white},   % choose the background color; you must add \usepackage{color} or \usepackage{xcolor}; should come as last argument
  basicstyle=\footnotesize,        % the size of the fonts that are used for the code
  breakatwhitespace=false,         % sets if automatic breaks should only happen at whitespace
  breaklines=true,                 % sets automatic line breaking
  captionpos=b,                    % sets the caption-position to bottom
  commentstyle=\color{mygreen},    % comment style
  deletekeywords={...},            % if you want to delete keywords from the given language
  escapeinside={\%*}{*)},          % if you want to add LaTeX within your code
  extendedchars=true,              % lets you use non-ASCII characters; for 8-bits encodings only, does not work with UTF-8
  firstnumber=1,                   % start line enumeration with line 1000
  frame=single,	                 % adds a frame around the code
  keepspaces=true,                 % keeps spaces in text, useful for keeping indentation of code (possibly needs columns=flexible)
  keywordstyle=\color{blue},       % keyword style
  %language=Octave,                % the language of the code
  morekeywords={*,...},            % if you want to add more keywords to the set
  numbers=left,                    % where to put the line-numbers; possible values are (none, left, right)
  numbersep=5pt,                   % how far the line-numbers are from the code
  numberstyle=\tiny\color{mygray}, % the style that is used for the line-numbers
  rulecolor=\color{black},         % if not set, the frame-color may be changed on line-breaks within not-black text (e.g. comments (green here))
  showspaces=false,                % show spaces everywhere adding particular underscores; it overrides 'showstringspaces'
  showstringspaces=false,          % underline spaces within strings only
  showtabs=false,                  % show tabs within strings adding particular underscores
  stepnumber=1,                    % the step between two line-numbers. If it's 1, each line will be numbered
  stringstyle=\color{mymauve},     % string literal style
  tabsize=2,	                   % sets default tabsize to 2 spaces
  title=\lstname                   % show the filename of files included with \lstinputlisting; also try caption instead of title
}

