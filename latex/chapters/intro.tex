% ::setlocal makeprg=cd\ latex\ &&\ pdflatex\ -interaction=batchmode\ main.tex\ &&\ xdg-open\ main.pdf\ &&\ exit

Newtonian mechanics is built upon the concept of absolute time and space.
Once the concept of \textit{inertial frame} is introduced, physics can be
done on a space described by Euclidean geometry.
For inertial observers, free particles (particles on which no net forces are
acting) move in a straight line, which is the shortest distance between two
points in a three-dimensional space, measured as:

\begin{equation}
    \Delta s^2 = \Delta x^2 + \Delta y^2 + \Delta z^2 \, .
    \label{eq:euclide}
\end{equation}

On the other hand, time is \textit{just} seen as a parameter, common to every 
inertial frame, that can be used to parametrize the trajectory and to determine
the particle velocity and acceleration.

With the appearance of Maxwell's Equations it became clear that what they 
predicted (the speed of light being constant in every inertial frame) was in
contrast with the description of our space given by Newtonian Mechanics, where
the speed of anything changes with respect to the chosen inertial frame.
Between Maxwell's Equations and Newtonian mechanics Einstein chose to modify 
the latter and wrote his two postulates for the theory of Special Relativity:

\begin{itemize}
    \item The laws of physics are invariant (identical) in all inertial frames
        of reference;
    \item The speed of light in vacuum,
        $c = \num{299792458} \unit[per-mode = symbol]{\meter\per\second}$,
        is the same for all observers, regardless of the motion of light source
        or observer.
\end{itemize}

The postulates may or may not be intuitive, but simple observations based on
them bring us to abandon the idea of absolute space and time and to introduce
the concept of \textit{spacetime}, together with a new way of measuring
distances

\begin{equation}
    \Delta s^2 = - c^2 \Delta t^2 + \Delta x^2 + \Delta y^2 + \Delta z^2 \, .
    \label{eq:Minkowski}
\end{equation}

In special relativity distances between events measured in this way are the same
for every observer in every possible inertial frame.

The appearance of time in a formula that is supposed to give us the distance
between two events is surely destabilizing at first, but geometry teaches us
that fixing the way we calculate $\Delta s^2$, more properly referred to as the
\textit{line element} $\mathrm{d}s^2$, is enough to describe the geometry of
the space that we are using.
Since eq. \ref{eq:Minkowski} is different from eq. \ref{eq:euclide},
in
particular there is a minus sign in front of $\Delta t^2$, we moved away from
the familiar three-dimensional Euclidean geometry and are now in
four-dimensional spacetime, usually referred to as \textit{flat spacetime} or
\textit{Minkowski space}.

This new geometry allowed for a reformulation of Maxwell's Equations and
brought (and explained) phenomena like time dilation, length contraction and the
relativity of simultaneity.
The last one in particular, the concept that the simultaneity of two events
depends on the frame of reference, poses a threat to the \textit{force} of
gravity.
Up until this point gravity was defined as the instantaneous force $F_{12}$
acting on a mass $m_1$ at time $t$ due to a second mass $m_2$:

\begin{equation}
    F_{12} = G \frac{m_1 m_2}{|r_1(t) - r_2(t)|^2}
    \label{eq:force_of_gravity}
\end{equation}

The adjective \textit{instantaneous} in a theory where nothing can travel
faster than the speed of light should already raise some concern.
But looking at $r_1(t)$ and $r_2(t)$ in eq. \ref{eq:force_of_gravity}, that are
supposed to indicate the positions of the masses in the same instant of time,
makes it even clearer that the force $F_{12}$ cannot be the same in all
frames of reference.

Solving this issue gave birth to the theory of general relativity, where a mass
is not a source of gravitational force anymore, but is responsible for
bending the four-dimensional spacetime itself.
This implies that when we observe a particle deviating its trajectory from a
straight line in the presence of a massive object, it is not because of a force
acting on it.
In fact, we can consider the particle free and moving from point A to point B
along the shortest path, it is just that in the curved surface bent by the mass
the shortest path is not a straight line.

While this concept may not enhance our intuitive understanding, the
implications and the mathematical formalism required to articulate the theory
are even more challenging.
If the presence of mass distorts the space we work in, changing the line
element $\mathrm{d}s^2$ is therefore necessary.
The details of the theories, particularly the Einstein field equations, that
describe this distortion and allow us to evaluate the new $\mathrm{d}s$ from a
give distribution of mass are beyond the scope of this thesis.
Our focus will be on evaluating the observable effects, given the line
element.

More specifically, we will study one of the simplest curved spacetimes that
general relativity offers: a stationary, spherically symmetric spacetime.
This geometry arises when the source of curvature is a spherical star but can
also exist in the absence of any source. The latter case is associated with the
presence of a black hole, with the corresponding solutions referred to as
\textit{black hole solutions}.
It is one of the simplest because of the many symmetries that presents and,
luckily, is also one of the most useful.

The line element of what is more commonly know as the \Sh geometry is

\begin{equation}
    \mathrm{d}s^2 = - \left(1 - \frac{2 G M}{c^2 r} \right) (c \mathrm{d}t)^2
    + \left(1 - \frac{2 G M}{c^2 r} \right)^{-1} \mathrm{d}r^2
    + r^2 (\mathrm{d}\theta^2 + \sin^2 \theta \mathrm{d}\phi^2)
    \label{eq:Sh}
\end{equation}

expressed in spherical coordinates centered in the mass responsible for bending
the space.

To explore this geometry, we will examine the possible trajectories of both a
free particle and a photon. By analyzing the paths of particles with varying
energies and directions as they move around the massive object, we can gain
insight into some of the implications of the \Sh metric.


%% qualcosa sul secondo capitolo?
