% ::setlocal makeprg=cd\ latex\ &&\ pdflatex\ -interaction=batchmode\ main.tex\ &&\ xdg-open\ main.pdf\ &&\ exit

In this thesis, the primary objective was to study the \Sh metric and explore
its consequences through the analysis of geodesic equations, which describe the
motion of test particles and light rays within a given metric.

The first chapter introduced the \Sh metric and its properties, serving as a
foundation for the second one.
In particular two effective potentials were derived (one for massive particle
and one for light rays), which allowed us to study the orbits in a
Newtonian-like fashion.
For the massive case we explored specific scenarios such as radial infalls and
circular orbits, because they could be solved analytically.
We also studied the precession of the perihelion of a test particle, compared
the results with the Newtonian case, but in this case there is no analytical
solution, so the integral was left for the second part.

The second chapter was dedicated to documenting a program that numerically
integrates the equations of motion for a massive particle.
The analytical results of the previous chapter served as an immediate test for
the results of our simulations.
Once the program reached a satisfactory level of accuracy, we showed some of the
more complex infalls we could find by tweaking the initial conditions.
Finally, when we studied the precession, we acknowledged the limitations of the
integration techniques initially used, and proposed a more accurate method. 
We decided to rely on the derivative of the effective potential, thus an
effective force, and solved a lot of the issues that were emerged previously.

Despite the improvements, the new numerical model has limitations.
One potential enhancement for future work could be the implementation of an
adaptive step size method, which would enable more accurate and efficient
integration of the equations of motion, particularly in regions near the \Sh
radius.
The focus though was on building a solid tool to visualize some of the orbits
that general relativity has to offer and providing the tests to 
validate it, in this regard this goal was achieved.
