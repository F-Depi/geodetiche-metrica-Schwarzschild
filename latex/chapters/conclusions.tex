% ::setlocal makeprg=cd\ latex\ &&\ pdflatex\ -interaction=batchmode\ main.tex\ &&\ xdg-open\ main.pdf\ &&\ exit

In this thesis the primary focus was to study the \Sh metric and to understand
some of its consequences.
We did so by studying the so-called geodesic equations, which describe the
motion of a test particle or light ray in a given metric.

The first chapter introduced the \Sh metric and its properties, serving as a
foundation for the second one.
In particular two effective potentials were derived (one for massive particle an
on for light rays), which allowed us to study the orbits in a Newtonian-like
fashion.
For the massive case we explored specific   such as radial infalls and circular
orbits, because they could be solved analytically.
We also studied the precession of the perihelion of a test particle, compared
the results with the Newtonian case, but in this case there is no analytical
solution, so the integral was left for the second part.

The second chapter was dedicated to documenting a program that numerically
integrates the equations of motion for a massive particle.
The analytical results of the previous chapter served as an immediate for the
results of our simulations.
Once the program reached a satisfactory level of accuracy, we showed some of the
more complex infalls we could find by tweaking the initial conditions.
Finally, when we studied the precession, we acknowledged the limitations of the
integration techniques initially used, and proposed a more accurate method. 
We decided to rely on the derivative of the effective potential, thus an
effective force, and solved a lot of the issues that were emerged previously.

