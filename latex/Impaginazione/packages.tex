% generali
\usepackage[utf8]{inputenc}
\usepackage{geometry}
\usepackage[english]{babel}
\usepackage{csquotes}
\usepackage{graphicx}
\usepackage{listings}
\usepackage[svgnames]{xcolor}
\usepackage{comment}
\usepackage{setspace}
\usepackage{appendix}

%\usepackage[9-15]{pagesel}
% attenzione: la filigrana in prima pagina ha opacità massima con questa opzione attivata
% non mettendo la prima pagina comunque appare draft in background

%matematica
\usepackage{siunitx}
\usepackage{amsmath}
\usepackage{amssymb}
\usepackage{physics}
\usepackage{braket}

% prima pagina
\usepackage{frontespizio}
\usepackage{tikz}
\usepackage[pages=some]{background}

% linea sopra le pagine
\usepackage{fancyhdr}
\pagestyle{fancy}
\fancyhf{}
\fancyhead[LE]{\leftmark}
\fancyhead[RO]{\rightmark}
\fancyfoot[LE, RO]{\thepage}


% bibliografia
\usepackage[style=numeric]{biblatex}
\addbibresource{biblio.bib}

% abstract
\newenvironment{abstract}%
    {\cleardoublepage%
        \thispagestyle{empty}%
        \null \vfill\begin{center}%
            \bfseries \abstractname \end{center}}%
        {\vfill\null}

% indice analitico
\usepackage{imakeidx}
\makeindex

% simboli e notazioni 
\usepackage{nomencl}
\makenomenclature

%da caricare come ultimo pacchetto per evitare problemi
\usepackage{glossaries}
\makeglossaries
\newglossaryentry{set}{name={set},description={a collection of objects}}
\newglossaryentry{due}{name={second},description={second example}}
\newglossaryentry{emptyset}{name={\ensuremath{\emptyset}},description={the empty set}}
\longnewglossaryentry{fishage}{name={Fish Age}}{
    A common name for the Devonian geologic period spanning from the end of the Silurian Period to the beginning of the Carboniferous Period.
    
    This age was known for its remarkable variety of fish species.
}
\newglossaryentry{elite}{name={{é}lite},description={select group or class}}

% aggiunte da me
\usepackage{easybmat}
\usepackage{nicematrix}
\usepackage{tikz-3dplot}
\usepackage[outline]{contour} % glow around text
\newcommand{\Sh}[0]{Schwarzschild }
\newcommand{\Mi}[0]{Minkowski }
