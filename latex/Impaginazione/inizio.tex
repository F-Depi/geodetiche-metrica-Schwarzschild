% ::setlocal makeprg=cd\ latex\ &&\ pdflatex\ -interaction=batchmode\ main.tex\ &&\ xdg-open\ main.pdf\ &

\Universita{University of Trento}
\Dipartimento{Department of Physics}
\CorsoDiLaurea{Bachelor’s Degree in Physics }
\AnnoAccademico{Academic Year 2023--2024}
\Titolo{Geodesics in \Sh Metric}
\Relatore{Prof. Albino \textsc{Perego}}
\RelatoreLabel{Supervisor}
\CandidatoLabel{Graduate Student}

\Candidato{Federico \textsc{De Paoli}} 
\Matricola{227552}
\DataEsame{September 20, 2024}
\Logo{Impaginazione/logo_rosso.png}
\LogoWidth{4.5cm} %optional, default: 3cm
\LogoPosition{top}
\LogoSfondo{Impaginazione/logo_bn.png}
\opacitaSfondo{0.05}

\begin{titlepage}
    \newgeometry{left=3cm, right=3cm, bottom=2cm, top =3cm} 
    \pagestyle{empty}
    \makefrontpage
    \restoregeometry
\end{titlepage}

\thispagestyle{empty}
\frontmatter
%\pagestyle{plain}

%% dedica
%~ \newpage
%\null\vspace{\stretch{1}}
%\begin{flushright}
%    \textit{Placeholder}
%\end{flushright}
%\vspace{\stretch{2}}\null

%ringraziamenti
\chapter*{Acknowledgments}
I would like to thank my family and friends for their support and encouragement
during my studies and throughout the production of this thesis.
A special thanks to my dear friend Arrigo, whose persistent encouragement
motivated me to improve my English skills.

I also wish to thank my supervisor, Dr. Albino Perego, for his guidance
throughout this work.
His ability to find a topic that perfectly aligned with both my interests and
capabilities made this thesis possible.

\begin{abstract}
    The \Sh metric provides a straightforward and effective framework for a
    first approach at the theory of general relativity.
    This thesis examines the metric through the motion of test particles and
    photons.
    Analytical solutions are presented for phenomena such as gravitational
    redshift and radial infalls. The precession of orbits and the deflection of
    light are explored by deriving the corresponding equations. \\
    Equations of motion for test particles are solved in case of strong
    gravitational field with a fourth order Runge-Kutta method.
    Numerical solutions are confronted with analytical ones when possible and
    stability tests are performed by repeating the same simulations with
    different time step increments.
\end{abstract}

\tableofcontents

%\printglossaries
%\addcontentsline{toc}{chapter}{Glossary}
%
%\printnomenclature
%\addcontentsline{toc}{chapter}{Nomenclature list}

\mainmatter
\chapter*{Introduction}
\markboth{\MakeUppercase{Introduction}}{}
\addcontentsline{toc}{chapter}{Introduction}
% ::setlocal makeprg=cd\ latex\ &&\ pdflatex\ -interaction=batchmode\ main.tex\ &&\ xdg-open\ main.pdf
Studio delle geodetiche in metrica di \Sh{}.
Partendo da una delle più semplici geometrie dello spazio-tempo che offre la relatività generale, la metrica di \Sh{}, vengono studiati alcuni dei fenomeni più comuni ad essa associati dal punto di vista teorico.
% Lista delle cose che metto
Nella seconda parte vengono fatte delle simulazioni numeriche
% per? su? quante? vediamo..

gravitational physics


