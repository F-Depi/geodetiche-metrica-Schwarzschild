% ::setlocal makeprg=cd\ latex\ &&\ pdflatex\ -interaction=batchmode\ main.tex\ &&\ xdg-open\ main.pdf\ &&\ exit

\chapter{Geometrized Units} \label{ap:geometrized_units}

In general relativity the constants $G \simeq \num{6.67430e-11}
\unit{\newton\meter\squared\per\kilo\gram\squared}$ and 
$c = \num{299792458} \unit[per-mode = symbol]{\meter\per\second}$ appears quite
often, so it's useful to redefine our units of measurements to cancel them out.
Let's take the \Sh line element as an example

\begin{equation}
    \mathrm{d}s^2 = - \left(1 - \frac{2 G M}{c^2 r} \right) (c \mathrm{d}t)^2
    + \left(1 - \frac{2 G M}{c^2 r} \right)^{-1} \mathrm{d}r^2
    + r^2 (\mathrm{d}\theta^2 + \sin^2 \theta \mathrm{d}\phi^2)
    \, .
    \label{ap1:eq:1}
\end{equation}

Time always appears next to the speed of light, $ct$. This is effectively as if
we were measuring time with the distance that light can cover in t seconds.
%We can then substitute $\hat t = ct$ and get
%
%\begin{equation*}
%    \mathrm{d}s^2 = - \left(1 - \frac{2 G M}{c^2 r} \right) \mathrm{d} \hat t^2
%    + \left(1 - \frac{2 G M}{c^2 r} \right)^{-1} \mathrm{d}r^2
%    + r^2 (\mathrm{d}\theta^2 + \sin^2 \theta \mathrm{d}\phi^2)
%    \, .
%\end{equation*}

The mass $M$ is measured in \unit{\kilogram} in S.I. units.
If we multiply it by $G$ and divide by $c^2$ we get 

\begin{equation*}
    \left[\frac{GM}{c^2} \right] =
    \frac{\unit{\newton\meter\squared\per\kilo\gram\squared\kilo\gram}}
    {\unit{\meter\squared\per\second\squared}} = 
    \frac{\unit{\newton}}{\unit{\kilogram}} \unit{\second\squared}=
    \unit{\meter} \, .
\end{equation*}

So, in a less intuitive way, we can measure the mass as a distance too.

Substituting $\hat t = ct$ and $\hat M = \frac{GM}{c^2}$ in eq. \ref{ap1:eq:1}
gives


\begin{equation*}
    \mathrm{d}s^2 = - \left(1 - \frac{2 \hat M}{r} \right) \mathrm{d} \hat t^2
    + \left(1 - \frac{2 \hat M}{r} \right)^{-1} \mathrm{d}r^2
    + r^2 (\mathrm{d}\theta^2 + \sin^2 \theta \mathrm{d}\phi^2)
    \, .
\end{equation*}

In this way we went from a $\mathcal{LMT}$(Length Mass Time) units system, to
an $\mathcal{L}$ one.
Since it will be clearly said when this convention, loosely referred to as
$G = c = 1$, is in use we will omit the \textit{hat} in the new defined
variables.

To go back to $\mathcal{LMT}$ units we just need to substitute back
$t \rightarrow ct$ and $M \rightarrow \frac{GM}{c^2}$, being extra careful
on cases like the speed, that is derived from time and therefor inherits a
$1 / c$ factor.

Table \ref{ap:tab:geometrized} shows some typical masses values expressed in
geometrical units.

\begin{table}[h]
    \centering
    \begin{tabular}{c|c|c}
         & S.I. units & Geometrized units \\
         \hline
         Mass of the Earth &
         \num{5.97e24} \unit{\kilogram} &
         4.43 \unit{\milli\meter} \\
         \hline
         Mass of the Sun &
         \num{1.99e30} \unit{\kilogram}
         & 1.48 \unit{\kilo\meter} \\
         \hline
         M87 black hole
         & \num{6.5e9}$M_\odot$
         & 64.2 \unit{\astronomicalunit}
    \end{tabular}
    \caption{Some common masses of the universe expressed in unit of length.}
    \label{ap:tab:geometrized}
\end{table}
