% ::setlocal makeprg=cd\ latex\ &&\ pdflatex\ -interaction=batchmode\ main.tex\ &&\ xdg-open\ main.pdf\ &&\ exit
\section{Introduction}

\subsection{Why the \Sh Geometry}

Newtonian mechanics is built upon the concept of absolute time and space.
Once the concept of \textit{inertial frame} is well-defined, physics can be
done on a space described by Euclidean geometry.
Free particles (particles on which no forces are acting) move in a straight
line, which is the shortest distance between two points in a three-dimensional
space, measured as:

\begin{equation}
    \Delta s^2 = \Delta x^2 + \Delta y^2 + \Delta z^2 \, .
    \label{eq:euclide}
\end{equation}

On the other hand, time is \textit{just} seen as a parameter, common to every 
inertial frame, that can be used to determine the particle velocity and
acceleration.

With the appearance of Maxwell's Equations it became clear that what they 
predicted (the speed of light being constant in every inertial frame) was in
contrast with the description of our space given by Newtonian Mechanics, where
the speed of anything changes with respect to the inertial frame chosen.
Between Maxwell's Equations end Newtonian mechanics Einstein chose to modify 
the latter and wrote his two postulates for the theory of Special Relativity:

\begin{itemize}
    \item The laws of physics are invariant (identical) in all inertial frames
        of reference;
    \item The speed of light in vacuum,
        $c = \num{299792458} \unit[per-mode = symbol]{\meter\per\second}$,
        is the same for all observers, regardless of the motion of light source
        or observer.
\end{itemize}

The postulates may or may not be intuitive, but simple observations based on
them bring us to abandon the idea of absolute space and time and to introduce
the concept of \textit{spacetime}, together with a new way of measuring
distances

\begin{equation}
    \Delta s^2 = - c^2 \Delta t^2 + \Delta x^2 + \Delta y^2 + \Delta z^2 \, .
    \label{eq:Minkowski}
\end{equation}

In special relativity distances measured this way are the same for every
observer in every inertial frame possible.

The appearance of time in a formula that is supposed to give us the distance
between 2 objects is surely destabilizing at first, but geometry teaches us
that fixing the way we calculate $\Delta s^2$, more properly referred to as the
\textit{line element} $\mathrm{d}s^2$, it's enough to describe the geometry of
the space that we are using.
Since eq. \ref{eq:Minkowski} is different from eq. \ref{eq:euclide}, in
particular there is a minus sign in front of $\Delta t^2$, we moved away from
the familiar three-dimensional Euclidean geometry and are now in
four-dimensional spacetime, usually referred to as \textit{flat spacetime} or
\textit{Minkowski space}.

This new geometry allowed for a reformulation of Maxwell's Equations and
brought (and explained) phenomena like time dilation, length contraction and the
relativity of simultaneity.
The last one in particular, the concept that the simultaneity of two events
depends on the frame of reference, poses a thread to the \textit{force} of
gravity.
Up until this point gravity was defined as the instantaneous force $F_{12}$
acting on a mass $m_1$ at time $t$ due to a second mass $m_2$:

\begin{equation}
    F_{12} = G \frac{m_1 m_2}{|r_1(t) - r_2(t)|^2}
    \label{eq:force_of_gravity}
\end{equation}

The adjective \textit{instantaneous} in a theory where nothing can travel
faster than the speed of light should already raise some concern.
But looking at $r_1(t)$ and $r_2(t)$ in eq. \ref{eq:force_of_gravity}, that are
supposed to indicate the positions of the masses in the same instant of time,
makes it even more clear that the force $F_{12}$ can't be the same in all
frames of reference.

Solving this issue gave birth to the theory of general relativity, where a mass
is not a source of gravitational force anymore, but it's responsible for
bending the four-dimensional spacetime itself.
This implies that when we observe a particle deviating its trajectory from a
straight line in the presence of a massive object, it's not because of a force
acting on it.
In fact, we can consider the particle free and moving from point A to point B
along the shortest path, it's just that in the curved surface bent by the mass
the shortest path is not a straight line.

While this concept may not enhance our intuitive understanding, the
implications and the mathematical formalism required to articulate the theory
are even more challenging.
If the presence of mass distorts the space we work in, changing in the line
element $\mathrm{d}s^2$ is then necessary.
The detail of the theories, particularly the Einstein field equations, that
describe this distortion and allow us to evaluate the new $\mathrm{d}s$ from a
give distribution of mass are beyond the scope of this thesis.
Our focus will be on evaluating the effect that are observable, given the line
element.

More specifically we will study one of the simplest curved spacetime that
general relativity has to offer: the geometry of empty space outside a
spherically symmetric source of curvature, for example, a spherical
star.
It's one of the simplest because of the many symmetries that presents and,
luckily, is also one of the most useful.

The line element of what is more commonly know as the \Sh geometry is

\begin{equation*}
    \mathrm{d}s^2 = - \left(1 - \frac{2 G M}{c^2 r} \right) (c \mathrm{d}t)^2
    + \left(1 - \frac{2 G M}{c^2 r} \right)^{-1} \mathrm{d}r^2
    + r^2 (\mathrm{d}\theta^2 + \sin^2 \theta \mathrm{d}\phi^2)
\end{equation*}

expressed in spherical coordinates centered in the mass responsible for bending
the space.




\newpage
\subsection{Notation and Formalism} \label{ssec:notation}

In the \textit{flat spacetime} we can introduce a coordinate basis for
four-vectors
\begin{equation}
    \mathbf{e_t} = (1,0,0,0), \quad
    \mathbf{e_x} = (0,1,0,0), \quad
    \mathbf{e_y} = (0,0,1,0), \quad
    \mathbf{e_z} = (0,0,0,1).
\end{equation}
The set 
$\{ \mathbf{e_t}, \mathbf{e_x}, \mathbf{e_y}, \mathbf{e_z} \}$, is often
referred to as $\{ \mathbf{e_0}, \mathbf{e_1}, \mathbf{e_2}, \mathbf{e_3} \}$.
Any four-vector $\textbf{a}$ can then be written as

\begin{equation}
    \textbf{a}
    = a^t \mathbf{e_t} + a^x \mathbf{e_x} + a^y \mathbf{e_y} + a^z \mathbf{e_z}
    = a^0 \mathbf{e_0} + a^1 \mathbf{e_1} + a^2 \mathbf{e_2} + a^3 \mathbf{e_3}
    \label{eq:a}
\end{equation}

where $(a_t,~a_x,~a_y,~a_z)$, or equivalently $(a_0,~a_1,~a_2,~a_3)$, are the
components of the four-vector.
Both notations will be used.

Another useful convention is to use Roman letters (usually $i$ or $j$) to refer
to indices 1, 2, 3 and Greek letters (usually $\mu$ or $\nu$) to refer to
indices 0, 1, 2, 3.
Using Einstein notation the expression in eq. \ref{eq:a}, can be rewritten
simply as $\textbf{a} = a^\mu \mathbf {e_\mu}$.
Other useful ways to specify the components of $\textbf{a}$ are

\begin{equation*}
    a^\mu = (a^t, a^x, a^y, a^z) \quad a^\mu = (a^t, a^i) \quad a^\mu
    = (a^t, \vec a)
\end{equation*}

where $\vec a = a^i e_i$ is the tree-dimensional vector $(a_x, a_y, a_z)$.

The length of the four-vector $\mathbf{a}$ must match the definition given with
the $\Delta s^2$ in \ref{eq:Minkowski}, it's useful to define the
\textit{metric} $\eta_{\nu \mu}$ so that

\begin{equation}
    \eta_{\nu \mu} = 
    \begin{array}{cc}
        \begin{pNiceMatrix}[first-row,first-col][columns-width = auto]
              & t & x & y & z \\
            t~~ & -1 & 0 & 0 & 0 \\  
            x~~ & 0 & 1 & 0 & 0 \\ 
            y~~ & 0 & 0 & 1 & 0 \\
            z~~ & 0 & 0 & 0 & 1 \\
        \end{pNiceMatrix} &
    \end{array}
    \quad \implies \quad
    \mathrm{d}s^2 = \eta_{\nu \mu} \mathrm{d}x^\nu \mathrm{d}x^\mu
\end{equation}

where a double sum is implied, and we rightfully notice that the minus sign has
appeared again under the $t$ component.
Now we can compactly write

\begin{equation}
    \mathbf{a} \cdot \mathbf{a} = \eta_{\mu \nu} \, a^\mu \, a^\nu
    = - (a^t)^2 + (a^x)^2 + (a^y)^2 + (a^z)^2
\end{equation}

Without any claim of rigorously demonstrating it, we can say that since this
scalar product is built from the line element $\mathrm{d}s^2$, it's the same
in every inertial frame one might choose. Quantities that have these properties
are \textit{invariant}.

When working in the \Sh geometry it's useful to adopt the \Sh coordinates,
spherical coordinates centered at the center of the mass $M$, and use
geometrized units, where $G = c = 1$.
The line element and the metric can be rewritten as

\begin{equation}
    \mathrm{d}s^2 = - \left(1 - \frac{2 M}{r} \right) (c \mathrm{d}t)^2
    + \left(1 - \frac{2 M}{r} \right)^{-1} \mathrm{d}r^2
    + r^2 (\mathrm{d}\theta^2 + \sin^2 \theta \mathrm{d}\phi^2)
    \label{eq:Sh_ds}
\end{equation}

\begin{equation}
    g_{\nu \mu} = 
    \begin{array}{cc}
        \begin{pNiceMatrix}[first-row,first-col][columns-width = auto]
              & t & r & \theta & \phi \\
            t~~ & - (1 - 2 M /r) & 0 & 0 & 0 \\  
            r~~ & 0 & (1 - 2 M /r)^{-1} & 0 & 0 \\ 
            \theta~~ & 0 & 0 & r^2 & 0 \\
            \phi~~ & 0 & 0 & 0 & r^2 \sin^2 \theta \\
        \end{pNiceMatrix} &
    \end{array}
    .
    \label{eq:Sh_g}
\end{equation}

It's worth noting that, given \ref{eq:Sh_ds}, the orthonormal frame indtroduced
becomes

\begin{equation}
    \mathbf{\hat e_t} = \left(1 - \frac{2M}{r}\right)^{-1/2} \mathbf{e_t}
\end{equation}

\newpage
\section{Conserved Quantities}

