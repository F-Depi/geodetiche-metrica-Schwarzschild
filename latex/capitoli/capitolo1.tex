% ::setlocal makeprg=cd\ latex\ &&\ pdflatex\ -interaction=batchmode\ main.tex\ &&\ xdg-open\ main.pdf\ &&\ exit
\section{Introduction}

\subsection{Why the \Sh{} Geometry}

Newtonian mechanics is built upon the concept of absolute time and space.
Once the concept of \textit{inertial frame} is well defined, physics can be
done on a space described by Euclidean geometry.
Free particles (particles on which no forces are acting) move in a straight
line, which is the shortest distance between two points in a three-dimensional
space, measured as:

\begin{equation}
    \Delta s^2 = \Delta x^2 + \Delta y^2 + \Delta z^2 \, .
    \label{eq:euclide}
\end{equation}

On the other hand, time is \textit{just} seen as a parameter, common to every 
inertial frame, that can be used to determine the particle velocity and
acceleration.

With the appearance of Maxwell's Equations it became clear that what they 
predicted (the speed of light being constant in every inertial frame) was in
contrast with the description of our space given by Newtonian Mechanics, where
the speed of anything changes with respect to the inertial frame chosen.
Between Maxwell's Equations end Newtonian mechanics Einstein chose to modify 
the latter and wrote his two postulates:

\begin{itemize}
    \item The laws of physics are invariant (identical) in all inertial frames
        of reference;
    \item The speed of light in vacuum,
        $c = \num{299792458} \unit[per-mode = symbol]{\meter\per\second}$,
        is the same for all observers, regardless of the motion of light source
        or observer.
\end{itemize}

The postulates may or may not be intuitive, but simple observations based on
them bring us to abandon the idea of absolute space and time and to introduce
the concept of \textit{spacetime}, together with a new way of measuring
distances

\begin{equation}
    \Delta s^2 = - c^2 \Delta t^2 + \Delta x^2 + \Delta y^2 + \Delta z^2 \, .
    \label{eq:Minkowski}
\end{equation}

The appearance of time in a formula that is supposed to give us the distance
between 2 objects is surely destabilizing at first, but geometry teaches us
that fixing the way we calculate $\Delta s^2$, more properly referred to as the
\textit{line element} $\mathrm{d}s^2$, it's enough to describe the geometry of
the space that we are using.
Since eq. \ref{eq:Minkowski} is different from eq. \ref{eq:euclide}, in
particular there is a minus sign in front of $\Delta t^2$, we moved away from
the familiar three-dimensional Euclidean geometry and are now in
four-dimensional spacetime, usually referred to as \textit{flat spacetime} or
\textit{Minkowski space}.

This new geometry allowed for a reformulation of Maxwell's Equations and
brought (and explained) fenomena like time dilation, length contraction and the
relativity of simultaneity.
The last one in particular, the concept that the simultaneity of two events
depends on the frame of reference, poses a thread to the \textit{force} of
gravity.
Up until this point gravity was defined as the istantaneous force $F_{12}$
acting on a mass $m_1$ at time $t$ due to a second mass $m_2$:

\begin{equation}
    F_{12} = G \frac{m_1 m_2}{|r_1(t) - r_2(t)|^2}
    \label{eq:force_of_gravity}
\end{equation}

The abjective \textit{istantaneous} in a theory where nothing can travel faster
then the speed of light should already raise some concern.
But looking at $r_1(t)$ and $r_2(t)$ in eq. \ref{eq:force_of_gravity}, that are
supposed to indicate the positions of the masses in the same istant of time,
makes it even more clear that the force $F_{12}$ can't be the same in all
frames of reference.

Solving this issue gave birth to the theory of general relativity, where a mass
is not a source of gravitational force, but it's responsible for bending the
four-dimensional spacetime.
This implies that when we observe a particle deviating its trajectory from a
straight line in the presence of a massive object, it's not because of a force
acting on it.
In fact we can consider the particle free and moving from point A to point B
along the shortest path, it's just that in a curved surface the shortest path
is not a straight line.

These idea surely doesn't make things more intuitive to imagine, but the
implication on the formalism that is needed to discuss the theory are even
worst.
Bending the space we are working on because of the presence of a mass means
that the line element $\mathrm{d}s^2$ must change.
The implication of this and the theories that lie behind (Einstein field
equations most notably) are behind the scope of this thesis.

Here we will focus on a specific study case: the geometry of empty space
outside a spherically symmetric source of curvature, for example, a spherical
star.
This is one of the simplest curved spacetimes of general relativity because of
the many symmetries that presents and, luckily, is also one of the most useful.
The line element of what is called \Sh{} geometry is

\begin{equation}
    \mathrm{d}s^2 = - \left(1 - \frac{2 G M}{c^2 r} \right) (c \mathrm{d}t)^2
    + \left(1 - \frac{2 G M}{c^2 r} \right)^{-1} \mathrm{d}r^2
    + r^2 (\mathrm{d}\theta^2 + \sin^2 \theta \mathrm{d}\phi^2)
    \label{eq:Sh_ds}
\end{equation}

expressed in spherical coordinates centered in the center of the mass.


\subsection{Notation and formalisms}
\label{ssec:notation}
In the \textit{flat spacetime}, chosen a particular inertial frame, we
introduce a basis for four-vectors
$\{ \mathbf{e_t}, \mathbf{e_x}, \mathbf{e_y}, \mathbf{e_z} \}$, or equivalently
$\{ \mathbf{e_0}, \mathbf{r_r}, \mathbf{e_2}, \mathbf{e_3} \}$, of unit length.
Any four-vector $\textbf{a}$ can then be written as

\begin{equation}
    \textbf{a}
    = a^t \mathbf{e_r} + a^x \mathbf{e_x} + a^y \mathbf{e_y} + a^z \mathbf{e_z}
    = a^0 \mathbf{e_0} + a^1 \mathbf{e_1} + a^2 \mathbf{e_2} + a^3 \mathbf{e_3}
    \label{eq:a}
\end{equation}

where $(a_t,~a_x,~a_y,~a_z)$, or $(a_0,~a_1,~a_2,~a_3)$, are the components of
the four-vector.
Both notations will be used.

Another useful convention is to use Roman letters (usually $i$ or $j$) to refer
to indices 1, 2, 3 and Greek letters (usually $\mu$ or $\nu$) to refer to
indices 0, 1, 2, 3.
Using Einstein notation the expression in eq. \ref{eq:a}, can be rewritten
simply as $\textbf{a} = a^\mu e_\mu$.
Other useful ways to specify the components of $\textbf{a}$ are

\begin{equation}
    a^\mu = (a^t, a^x, a^y, a^z) \quad a^\mu = (a^t, a^i) \quad a^\mu
    = (a^t, \vec a)
    \label{eq:a_notation}
\end{equation}

where $\vec a = a^i e_i$ is the tree-dimensional vector $(e_1, e_2, e_3)$









\section{Conserved Quantities}
\Glspl{set}

