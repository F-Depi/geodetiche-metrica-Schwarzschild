% ::setlocal makeprg=cd\ latex\ &&\ pdflatex\ -interaction=batchmode\ main.tex\ &&\ xdg-open\ main.pdf
\section{Why the \Sh{} Geometry}

Newtonian mechanics is built upon the concept of absolute time and space.
Once the concept of \textit{inertial frame} is well defined physics can be done on a space described by Euclidean geometry.
Free particles (particles on which no forces are acting) move in a straight line, which is the shortest distance between two points in a three-dimensional space, measured as:

\begin{equation}
    \Delta s^2 = \Delta x^2 + \Delta y^2 + \Delta z^2 \, .
    \label{eq:euclide}
\end{equation}

On the other end, time is \textit{just} seen as a parameter, common to every inertial frame,
that can be used to determine the particle velocity and acceleration.

With the appearance of Maxwell's Equations it became clear that what they predicted (the speed of light,
being constant in every inertial frame) was in contrast with the description of our space given by Newtonian Mechanics,
where the speed of anything changes with respect to the inertial frame chosen.
Between Maxwell's Equations end Newtonian mechanics Einstein chose to modify the latter and wrote his two postulates:

\begin{itemize}
    \item The laws of physics are invariant (identical) in all inertial frames of reference;
    \item The speed of light in vacuum, $c = \num{299792458} \unit[per-mode = symbol]{\meter\per\second}$, is the same for all observers, regardless of the motion of light source or observer.
\end{itemize}

The postulates may or may not be intuitive, but simple observations based on them bring us to abandon the idea
of absolute space and time and to introduce the concept of \textit{spacetime},
together with a new way of measuring distances

\begin{equation}
    \Delta s^2 = - c^2 \Delta t^2 + \Delta x^2 + \Delta y^2 + \Delta z^2 \, .
    \label{eq:Minkowski}
\end{equation}

The appearance of time in a formula that is supposed to give us the distance between 2 objects is surely destabilizing at first,
but geometry teaches us that fixing the way we calculate $\Delta s^2$,
more properly referred to as the \textit{line element} $\mathrm{d}s^2$,
it's enough to describe the geometry of the space that we are using.
Since eq. \ref{eq:Minkowski} is different from eq. \ref{eq:euclide},
in particular there is a minus sign in front of $\Delta t^2$,
we moved away from the familiar three-dimensional Euclidean geometry and are now in four-dimensional spacetime,
usually referred to as \textit{flat spacetime} or \textit{Minkowski space}.











\section{Conserved Quantities}
\Glspl{set}

